\documentclass[tikz,border=5pt]{standalone}

% PolyGraph Benchmark icon only

\usepackage{xcolor}
\usepackage{tikz}
\usetikzlibrary{calc,positioning}

% Palettes (copied from main for consistency)
\newcommand{\SelectPalette}[1]{\csname DefinePalette#1\endcsname}
\newcommand{\DefinePaletteSlateTealOrange}{%
  \definecolor{vertexFill}{HTML}{000000}
  \definecolor{vertexStroke}{HTML}{000000}
  \definecolor{vertexText}{HTML}{FFFFFF}
  \definecolor{edgeAColor}{HTML}{334155}
  \definecolor{edgeBColor}{HTML}{14B8A6}
  \definecolor{edgeCColor}{HTML}{F97316}
  \definecolor{labelColor}{HTML}{2C3E50}
}
\newcommand{\DefinePaletteIndigoEmeraldAmber}{%
  \definecolor{vertexFill}{HTML}{000000}
  \definecolor{vertexStroke}{HTML}{000000}
  \definecolor{vertexText}{HTML}{FFFFFF}
  \definecolor{edgeAColor}{HTML}{4F46E5}
  \definecolor{edgeBColor}{HTML}{10B981}
  \definecolor{edgeCColor}{HTML}{F59E0B}
  \definecolor{labelColor}{HTML}{2C3E50}
}

% Grayscale trio
\newcommand{\DefinePaletteGrayTrio}{%
  \definecolor{vertexFill}{HTML}{000000}
  \definecolor{vertexStroke}{HTML}{000000}
  \definecolor{vertexText}{HTML}{FFFFFF}
  \definecolor{edgeAColor}{HTML}{555555}
  \definecolor{edgeBColor}{HTML}{888888}
  \definecolor{edgeCColor}{HTML}{BBBBBB}
  \definecolor{labelColor}{HTML}{444444}
}

% Nord palettes (Aurora accents)
\newcommand{\DefinePaletteNordLight}{%
  % Unified trio mapping: A=P, C=G, B=B
  \definecolor{vertexFill}{HTML}{000000}
  \definecolor{vertexStroke}{HTML}{000000}
  \definecolor{vertexText}{HTML}{FFFFFF}
  \definecolor{edgeAColor}{HTML}{CBD3E1}% P
  \definecolor{edgeCColor}{HTML}{7187AD}% G
  \definecolor{edgeBColor}{HTML}{3C4C68}% B
  \definecolor{labelColor}{HTML}{3C4C68}
}
\newcommand{\DefinePaletteNordDark}{%
  % Unified trio mapping: A=P, C=G, B=B
  \definecolor{vertexFill}{HTML}{000000}
  \definecolor{vertexStroke}{HTML}{000000}
  \definecolor{vertexText}{HTML}{FFFFFF}
  \definecolor{edgeAColor}{HTML}{CBD3E1}% P
  \definecolor{edgeCColor}{HTML}{7187AD}% G
  \definecolor{edgeBColor}{HTML}{3C4C68}% B
  \definecolor{labelColor}{HTML}{CBD3E1}
}

% Theme selector (Light/Dark) to unify with other files
\newcommand{\SelectTheme}[1]{\csname DefineTheme#1\endcsname}
\newcommand{\ApplyThemeBaseColorsLight}{\definecolor{vertexFill}{HTML}{3C4C68}\definecolor{vertexStroke}{HTML}{3C4C68}\definecolor{vertexText}{HTML}{FFFFFF}\definecolor{labelColor}{HTML}{3C4C68}}
\newcommand{\ApplyThemeBaseColorsDark}{\definecolor{vertexFill}{HTML}{CBD3E1}\definecolor{vertexStroke}{HTML}{CBD3E1}\definecolor{vertexText}{HTML}{0F172A}\definecolor{labelColor}{HTML}{CBD3E1}}
\newcommand{\DefineThemeLight}{% Only base colors
  \ApplyThemeBaseColorsLight
}
\newcommand{\DefineThemeDark}{% Only base colors
  \ApplyThemeBaseColorsDark
}
% Select theme; can be overridden via \\THEME and optional \\PALETTE
\makeatletter
\newcommand{\DefaultPalette}{NordLight}
\@ifundefined{THEME}{\def\THEME{Light}\SelectTheme{Light}}{\SelectTheme{\THEME}}
% Apply default palette first, then override if provided
\SelectPalette{\DefaultPalette}
\@ifundefined{PALETTE}{}{\SelectPalette{\PALETTE}\csname ApplyThemeBaseColors\THEME\endcsname}
\makeatother

% Geometry
\newcommand{\VertexSize}{1.98mm}
\newcommand{\HexagonRadius}{1.0cm}
\newcommand{\EdgeThickness}{2.4pt}
% Slightly thinner edges for the standalone icon
\newcommand{\IconEdgeThickness}{0.85*\EdgeThickness}
\newcommand{\EdgeCurve}{18}
\newcommand{\EdgeCurveDouble}{12}

\tikzset{vertex/.style={circle, draw=vertexStroke, fill=vertexFill, text=vertexText, minimum size=\VertexSize, inner sep=0pt, line width=\EdgeThickness, line join=round}}

% Offset/curved helpers
\newcommand{\DrawOffsetEdge}[4]{%
  \pgfmathanglebetweenpoints{\pgfpointanchor{#1}{center}}{\pgfpointanchor{#2}{center}}%
  \let\AngleTmp\pgfmathresult%
  \begin{scope}[rotate around={\AngleTmp:(#1)}]
    \draw[#4] ($ (#1) + (0,#3) $) -- ($ (#2) + (0,#3) $);
  \end{scope}%
}
\newcommand{\DrawCurvedEdgeL}[4]{%
  \pgfmathanglebetweenpoints{\pgfpointanchor{#1}{center}}{\pgfpointanchor{#2}{center}}%
  \let\AngleTmp\pgfmathresult%
  \begin{scope}[rotate around={\AngleTmp:(#1)}]
    \draw[#4] ($ (#1) + (0,#3) $) to[bend left=\EdgeCurve] ($ (#2) + (0,#3) $);
  \end{scope}%
}
\newcommand{\DrawCurvedEdgeR}[4]{%
  \pgfmathanglebetweenpoints{\pgfpointanchor{#1}{center}}{\pgfpointanchor{#2}{center}}%
  \let\AngleTmp\pgfmathresult%
  \begin{scope}[rotate around={\AngleTmp:(#1)}]
    \draw[#4] ($ (#1) + (0,#3) $) to[bend right=\EdgeCurve] ($ (#2) + (0,#3) $);
  \end{scope}%
}
\newcommand{\DrawCurvedEdgeLDouble}[4]{%
  \pgfmathanglebetweenpoints{\pgfpointanchor{#1}{center}}{\pgfpointanchor{#2}{center}}%
  \let\AngleTmp\pgfmathresult%
  \begin{scope}[rotate around={\AngleTmp:(#1)}]
    \draw[#4] ($ (#1) + (0,#3) $) to[bend left=\EdgeCurveDouble] ($ (#2) + (0,#3) $);
  \end{scope}%
}
\newcommand{\DrawCurvedEdgeRDouble}[4]{%
  \pgfmathanglebetweenpoints{\pgfpointanchor{#1}{center}}{\pgfpointanchor{#2}{center}}%
  \let\AngleTmp\pgfmathresult%
  \begin{scope}[rotate around={\AngleTmp:(#1)}]
    \draw[#4] ($ (#1) + (0,#3) $) to[bend right=\EdgeCurveDouble] ($ (#2) + (0,#3) $);
  \end{scope}%
}

\begin{document}
\begin{tikzpicture}[scale=1]
  % Coordinates
  \path (0,0) coordinate (c);
  \foreach \i [evaluate=\i as \ang using 90 + 60*(\i-1)] in {1,...,6} {\path (\ang:\HexagonRadius) coordinate (v\i);} 

  % Lane drawers
  \newcommand{\DrawRedEdge}[2]{\DrawOffsetEdge{#1}{#2}{0pt}{draw=edgeAColor, line width=\IconEdgeThickness, line cap=round, line join=round}}
  \newcommand{\DrawGreenEdge}[2]{\DrawOffsetEdge{#1}{#2}{0pt}{draw=edgeBColor, line width=\IconEdgeThickness, line cap=round, line join=round}}
  \newcommand{\DrawBlueEdge}[2]{\DrawOffsetEdge{#1}{#2}{0pt}{draw=edgeCColor, line width=\IconEdgeThickness, line cap=round, line join=round}}
  \newcommand{\DrawRedEdgeCurveL}[2]{\DrawCurvedEdgeL{#1}{#2}{0pt}{draw=edgeAColor, line width=\IconEdgeThickness, line cap=round, line join=round}}
  \newcommand{\DrawGreenEdgeCurveL}[2]{\DrawCurvedEdgeL{#1}{#2}{0pt}{draw=edgeBColor, line width=\IconEdgeThickness, line cap=round, line join=round}}
  \newcommand{\DrawBlueEdgeCurveR}[2]{\DrawCurvedEdgeR{#1}{#2}{0pt}{draw=edgeCColor, line width=\IconEdgeThickness, line cap=round, line join=round}}
  \newcommand{\DrawRedEdgeCurveLDouble}[2]{\DrawCurvedEdgeLDouble{#1}{#2}{0pt}{draw=edgeAColor, line width=\IconEdgeThickness, line cap=round, line join=round}}
  \newcommand{\DrawGreenEdgeCurveLDouble}[2]{\DrawCurvedEdgeLDouble{#1}{#2}{0pt}{draw=edgeBColor, line width=\IconEdgeThickness, line cap=round, line join=round}}
  \newcommand{\DrawGreenEdgeCurveRDouble}[2]{\DrawCurvedEdgeRDouble{#1}{#2}{0pt}{draw=edgeBColor, line width=\IconEdgeThickness, line cap=round, line join=round}}
  \newcommand{\DrawBlueEdgeCurveRDouble}[2]{\DrawCurvedEdgeRDouble{#1}{#2}{0pt}{draw=edgeCColor, line width=\IconEdgeThickness, line cap=round, line join=round}}

  % Icon edges (same as main file)
  \begin{scope}[scale=1]
    % Red
    \foreach \a/\b in {v1/c, c/v4} {\DrawRedEdgeCurveLDouble{\a}{\b};}
    \DrawRedEdgeCurveLDouble{c}{v6}
    \DrawRedEdge{v6}{v1}

    % Blue
    \foreach \a/\b in {v1/v2, v2/v3, v3/v4} {\DrawBlueEdge{\a}{\b};}
    \foreach \a/\b in {v4/v5, v5/c} {\DrawBlueEdgeCurveRDouble{\a}{\b};}
    \DrawBlueEdgeCurveR{v6}{v1}

    % Green
    \DrawGreenEdgeCurveRDouble{c}{v4}
    \DrawGreenEdgeCurveLDouble{v4}{v5}
    \DrawGreenEdgeCurveLDouble{v5}{c}
    \DrawGreenEdgeCurveRDouble{c}{v6}
    \DrawGreenEdgeCurveRDouble{v1}{c}
    \DrawGreenEdgeCurveL{v6}{v1}
  \end{scope}

  % Nodes
  \foreach \p in {v1,c,v4,v6,v1,v2,v3,v5} {\node[vertex] at (\p) {};}
\end{tikzpicture}
\end{document}
