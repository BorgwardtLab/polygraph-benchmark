\documentclass[tikz,border=5pt]{standalone}

% PolyGraph Benchmark logo (TikZ)
% - Hex icon + styled title
% - Tweak colors and layout in the "User Tweaks" section
% - Select palettes via \SelectPalette{Name}

\usepackage{xcolor}
\usepackage{tikz}
\usetikzlibrary{calc,positioning}

% ===== User Tweaks (edit here) =====
% Color palettes
% - Define palettes as commands named \DefinePalette<Name>
% - Each palette must set: vertexFill, vertexStroke, vertexText, edgeAColor, edgeBColor, edgeCColor, labelColor
% - To switch palettes, call \SelectPalette{Name}

% Palette selector (do not edit)
\newcommand{\SelectPalette}[1]{\csname DefinePalette#1\endcsname}

% --- Theme selector (Light/Dark) ---
% Maps a theme to a palette and vertex/text colors. Call \SelectTheme{Light|Dark}.
\newcommand{\SelectTheme}[1]{\csname DefineTheme#1\endcsname}

% Base-color appliers
\newcommand{\ApplyThemeBaseColorsLight}{%
  % Light theme: vertices use darkest shade
  \definecolor{vertexFill}{HTML}{3C4C68}
  \definecolor{vertexStroke}{HTML}{3C4C68}
  \definecolor{vertexText}{HTML}{FFFFFF}
  \definecolor{labelColor}{HTML}{3C4C68}
}
\newcommand{\ApplyThemeBaseColorsDark}{%
  % Dark theme: vertices use lightest shade
  \definecolor{vertexFill}{HTML}{CBD3E1}
  \definecolor{vertexStroke}{HTML}{CBD3E1}
  \definecolor{vertexText}{HTML}{0F172A}
  \definecolor{labelColor}{HTML}{CBD3E1}
}

% Light theme (only sets base vertex/label colors)
\newcommand{\DefineThemeLight}{%
  \ApplyThemeBaseColorsLight
}

% Dark theme (only sets base vertex/label colors)
\newcommand{\DefineThemeDark}{%
  \ApplyThemeBaseColorsDark
}

% --- Built-in palettes ---
% A/B/C lanes map to: edgeAColor (lane A), edgeBColor (lane B), edgeCColor (lane C)

% 1) Indigo / Emerald / Amber (current default)
\newcommand{\DefinePaletteIndigoEmeraldAmber}{%
  \definecolor{vertexFill}{HTML}{000000}% black
  \definecolor{vertexStroke}{HTML}{000000}% black
  \definecolor{vertexText}{HTML}{FFFFFF}% white
  \definecolor{edgeAColor}{HTML}{4F46E5}% indigo 600
  \definecolor{edgeBColor}{HTML}{10B981}% emerald 500
  \definecolor{edgeCColor}{HTML}{F59E0B}% amber 500
  \definecolor{labelColor}{HTML}{2C3E50}% slate-like
}

% 2) Blue / Pink / Lime
\newcommand{\DefinePaletteBluePinkLime}{%
  \definecolor{vertexFill}{HTML}{000000}
  \definecolor{vertexStroke}{HTML}{000000}
  \definecolor{vertexText}{HTML}{FFFFFF}
  \definecolor{edgeAColor}{HTML}{2563EB}% blue 600
  \definecolor{edgeBColor}{HTML}{EC4899}% pink 500
  \definecolor{edgeCColor}{HTML}{84CC16}% lime 500
  \definecolor{labelColor}{HTML}{2C3E50}
}

% 3) Slate / Teal / Orange
\newcommand{\DefinePaletteSlateTealOrange}{%
  \definecolor{vertexFill}{HTML}{000000}
  \definecolor{vertexStroke}{HTML}{000000}
  \definecolor{vertexText}{HTML}{FFFFFF}
  \definecolor{edgeAColor}{HTML}{334155}% slate 700
  \definecolor{edgeBColor}{HTML}{14B8A6}% teal 500
  \definecolor{edgeCColor}{HTML}{F97316}% orange 500
  \definecolor{labelColor}{HTML}{2C3E50}
}

% 4) Navy / Gold / Crimson
\newcommand{\DefinePaletteNavyGoldCrimson}{%
  \definecolor{vertexFill}{HTML}{000000}
  \definecolor{vertexStroke}{HTML}{000000}
  \definecolor{vertexText}{HTML}{FFFFFF}
  \definecolor{edgeAColor}{HTML}{1E3A8A}% navy 800
  \definecolor{edgeBColor}{HTML}{D4AF37}% metallic gold
  \definecolor{edgeCColor}{HTML}{DC2626}% crimson 600
  \definecolor{labelColor}{HTML}{2C3E50}
}

% 5) Gray Trio (monochrome variants)
\newcommand{\DefinePaletteGrayTrio}{%
  \definecolor{vertexFill}{HTML}{000000}
  \definecolor{vertexStroke}{HTML}{000000}
  \definecolor{vertexText}{HTML}{FFFFFF}
  % Neutral grayscale trio
  \definecolor{edgeAColor}{HTML}{555555}% dark gray (lighter)
  \definecolor{edgeBColor}{HTML}{888888}% medium gray
  \definecolor{edgeCColor}{HTML}{BBBBBB}% light gray
  \definecolor{labelColor}{HTML}{444444}
}

% 6) Colorblind-safe (Okabe–Ito subset)
\newcommand{\DefinePaletteColorblindSafe}{%
  \definecolor{vertexFill}{HTML}{000000}
  \definecolor{vertexStroke}{HTML}{000000}
  \definecolor{vertexText}{HTML}{FFFFFF}
  \definecolor{edgeAColor}{HTML}{0072B2}% blue
  \definecolor{edgeBColor}{HTML}{009E73}% bluish green
  \definecolor{edgeCColor}{HTML}{D55E00}% vermillion
  \definecolor{labelColor}{HTML}{2C3E50}
}

% 7) Nord palette (Aurora accents). Two variants for light/dark contexts
%    Mapping: A=aurora red, B=aurora green, C=frost blue
%    Use with THEME=Light/\ApplyThemeBaseColorsLight or THEME=Dark/\ApplyThemeBaseColorsDark.
\newcommand{\DefinePaletteNordLight}{%
  % Unified trio mapping: A=P, C=G, B=B
  \definecolor{vertexFill}{HTML}{000000}
  \definecolor{vertexStroke}{HTML}{000000}
  \definecolor{vertexText}{HTML}{FFFFFF}
  \definecolor{edgeAColor}{HTML}{CBD3E1}% P
  \definecolor{edgeCColor}{HTML}{7187AD}% G
  \definecolor{edgeBColor}{HTML}{3C4C68}% B
  \definecolor{labelColor}{HTML}{3C4C68}
}
\newcommand{\DefinePaletteNordDark}{%
  % Unified trio mapping: A=P, C=G, B=B
  \definecolor{vertexFill}{HTML}{000000}
  \definecolor{vertexStroke}{HTML}{000000}
  \definecolor{vertexText}{HTML}{FFFFFF}
  \definecolor{edgeAColor}{HTML}{CBD3E1}% P
  \definecolor{edgeCColor}{HTML}{7187AD}% G
  \definecolor{edgeBColor}{HTML}{3C4C68}% B
  \definecolor{labelColor}{HTML}{CBD3E1}
}

% Select theme; allow optional palette override via \PALETTE
\makeatletter
\@ifundefined{THEME}{\def\THEME{Light}}{}
\SelectTheme{\THEME}
% Default palette follows THEME; then re-apply theme base colors
\SelectPalette{Nord\THEME}
\csname ApplyThemeBaseColors\THEME\endcsname
% Optional override preserves theme base vertex/label colors
\@ifundefined{PALETTE}{}{\SelectPalette{\PALETTE}\csname ApplyThemeBaseColors\THEME\endcsname}
\makeatother

% Geometry & sizing
\newcommand{\VertexSize}{1.98mm}                 % Node diameter (+10% from 1.8mm)
\newcommand{\HexagonRadius}{1.0cm}            % Distance from center to each corner (50% smaller)
\newcommand{\EdgeThickness}{2.4pt}            % Base edge thickness (20% thinner)
\newcommand{\ParallelBend}{26}                % Degrees for left/right parallel bends
\newcommand{\EdgeOffset}{1.0mm}               % Perpendicular offset for parallel straight edges (smaller for main logo)
\newcommand{\EdgeCurve}{18}                   % Bend amount (degrees) for curved lanes (triple-lane outer)
\newcommand{\EdgeCurveDouble}{12}             % Softer bend for double-lane pairs

% ===== Glyph (letter) rendering configuration =====
% Match glyph height (4 units) to hexagon height (2*\HexagonRadius)
% This sets each glyph unit to half the hexagon radius
\pgfmathsetlengthmacro{\GlyphUnit}{0.5*\HexagonRadius}
\newcommand{\LogoTextFont}{\sffamily\bfseries\fontsize{40pt}{44pt}\selectfont}
\newcommand{\IconScale}{1.00}                 % Scale factor for the left icon only
\newcommand{\GlyphAdvance}{4}                 % X-advance (in units) per letter
\newcommand{\WordGap}{2}                      % (legacy) extra units between words
\newcommand{\PolyGraphGap}{2}                 % Extra units between "Poly" and "Graph" (more space after 'oly')
\newcommand{\GraphBenchGap}{4.5}                % Extra units between "Graph" and "Benchmark" (more spacing)
\newcommand{\TextStartX}{2cm}                 % Absolute shift to the right of the logo
\newcommand{\TextBaselineY}{-\HexagonRadius} % Vertical baseline (align with logo bottom)

% Scale for glyph letters in the title (relative to base glyph unit). Tune so
% hex letters are ~2x the height of the adjacent text letters.
\newcommand{\GlyphTextScale}{0.70}

% Fine-tunable gaps (in multiples of \GlyphUnit) between custom glyphs and text
\newcommand{\GapPToOly}{-0.75}                 % Gap between custom P and "oly"
\newcommand{\GapGToRaph}{0.2}                % Gap between custom G and "raph"
\newcommand{\GapBToEnchmark}{0.2}            % Gap between custom B and "enchmark"

% Node and edge styles
% Node style (change once here to affect all nodes)
\tikzset{vertex/.style={circle, draw=vertexStroke, fill=vertexFill, text=vertexText, minimum size=\VertexSize, inner sep=0pt, line width=\EdgeThickness, line join=round}}

% Label styles (unused by default)
%\tikzset{vertexLabel/.style={font=\footnotesize\bfseries, text=labelColor}}
%\tikzset{edgeLabel/.style={font=\scriptsize\bfseries, text=labelColor, fill=white, fill opacity=0.85, text opacity=1, inner sep=1pt, rounded corners=1pt}}
%\tikzset{edgeLabelStraight/.style={edgeLabel, sloped, above}}
%\tikzset{edgeLabelLeft/.style={edgeLabel, sloped}}
%\tikzset{edgeLabelRight/.style={edgeLabel, sloped, below}}

% Four edge types (choose per edge)
\tikzset{edgeA/.style={draw=edgeAColor, line width=\EdgeThickness}}
\tikzset{edgeB/.style={draw=edgeBColor, line width=\EdgeThickness, dashed}}
\tikzset{edgeC/.style={draw=edgeCColor, line width=\EdgeThickness, densely dotted}}
\tikzset{edgeTransparent/.style={draw opacity=0, line width=\EdgeThickness}}

% Helpers to create three parallel edges between two nodes
\tikzset{straightEdge/.style={}}
\tikzset{bendLeftEdge/.style={bend left=\ParallelBend}}
\tikzset{bendRightEdge/.style={bend right=\ParallelBend}}

% Draw one straight edge offset perpendicularly by a given amount
% Usage: \DrawOffsetEdge{nodeA}{nodeB}{offset}{style}
\newcommand{\DrawOffsetEdge}[4]{%
  \pgfmathanglebetweenpoints{\pgfpointanchor{#1}{center}}{\pgfpointanchor{#2}{center}}%
  \let\AngleTmp\pgfmathresult%
  \begin{scope}[rotate around={\AngleTmp:(#1)}]
    \draw[#4] ($ (#1) + (0,#3) $) -- ($ (#2) + (0,#3) $);
  \end{scope}%
}

% Curved variants (left/right bend) with perpendicular offset
\newcommand{\DrawCurvedEdgeL}[4]{%
  \pgfmathanglebetweenpoints{\pgfpointanchor{#1}{center}}{\pgfpointanchor{#2}{center}}%
  \let\AngleTmp\pgfmathresult%
  \begin{scope}[rotate around={\AngleTmp:(#1)}]
    \draw[#4] ($ (#1) + (0,#3) $) to[bend left=\EdgeCurve] ($ (#2) + (0,#3) $);
  \end{scope}%
}
\newcommand{\DrawCurvedEdgeR}[4]{%
  \pgfmathanglebetweenpoints{\pgfpointanchor{#1}{center}}{\pgfpointanchor{#2}{center}}%
  \let\AngleTmp\pgfmathresult%
  \begin{scope}[rotate around={\AngleTmp:(#1)}]
    \draw[#4] ($ (#1) + (0,#3) $) to[bend right=\EdgeCurve] ($ (#2) + (0,#3) $);
  \end{scope}%
}

% Curved variants specifically for double-lane pairs (softer bend)
\newcommand{\DrawCurvedEdgeLDouble}[4]{%
  \pgfmathanglebetweenpoints{\pgfpointanchor{#1}{center}}{\pgfpointanchor{#2}{center}}%
  \let\AngleTmp\pgfmathresult%
  \begin{scope}[rotate around={\AngleTmp:(#1)}]
    \draw[#4] ($ (#1) + (0,#3) $) to[bend left=\EdgeCurveDouble] ($ (#2) + (0,#3) $);
  \end{scope}%
}
\newcommand{\DrawCurvedEdgeRDouble}[4]{%
  \pgfmathanglebetweenpoints{\pgfpointanchor{#1}{center}}{\pgfpointanchor{#2}{center}}%
  \let\AngleTmp\pgfmathresult%
  \begin{scope}[rotate around={\AngleTmp:(#1)}]
    \draw[#4] ($ (#1) + (0,#3) $) to[bend right=\EdgeCurveDouble] ($ (#2) + (0,#3) $);
  \end{scope}%
}

% Draw three parallel straight edges (centered, plus/minus offset) with independent styles
% Usage: \TripleEdgesTypes{nodeA}{nodeB}{styleStraight}{styleOffsetPlus}{styleOffsetMinus}
\newcommand{\TripleEdgesTypes}[5]{%
  \DrawOffsetEdge{#1}{#2}{0pt}{#3}%
  \DrawOffsetEdge{#1}{#2}{\EdgeOffset}{#4}%
  \DrawOffsetEdge{#1}{#2}{-\EdgeOffset}{#5}%
}

% Convenience: use the same style for all three parallel edges
% Usage: \TripleEdges{nodeA}{nodeB}{style}
\newcommand{\TripleEdges}[3]{%
  \TripleEdgesTypes{#1}{#2}{#3}{#3}{#3}%
}

% ===== Letter (glyph) helpers =====
% Each letter macro expects a unique prefix name and an integer x-offset in glyph units.
% It uses a color-agnostic edge macro \UseEdge{nodeA}{nodeB} set by the caller.

% Default no-op; will be overridden per word with a color-specific drawer
\newcommand{\UseEdge}[2]{}

% Hex-based glyph setup: creates 6 outer vertices and a center, all as nodes
% Orientation matches the main logo (pointy-top hex). Height = 4 units.
% Vertices are named: <prefix>-v1..v6 (angles: 90,150,210,270,330,30) and center <prefix>-c
\newcommand{\HexGlyphSetup}[1]{%
  \coordinate (#1-c) at (0,2);
  \foreach \k/\ang in {1/90,2/150,3/210,4/270,5/330,6/30} {\path ($(#1-c)+(\ang:2)$) coordinate (#1-v\k);}%
}

% Helper: place a coordinate and a node
\newcommand{\GlyphPoint}[3]{% prefix, x, y (grid coords)
  \coordinate (#1-#2-#3) at (#2,#3);\node[vertex] at (#1-#2-#3) {};%
}
% Helper: draw one segment using the active color macro
\newcommand{\Seg}[3]{% prefix, (x1,y1)--(x2,y2)
  \UseEdge{#1-#2}{#1-#3}%
}

% Letter P
\newcommand{\LetterP}[2]{% prefix, baseX (three-stroke P from hex outline)
  \begin{scope}[x=\GlyphUnit,y=\GlyphUnit,shift={(#2,0)}]
    \HexGlyphSetup{#1}
    % Strokes: left vertical, top, mid
    \UseEdge{#1-v2}{#1-v3}% left side
    \UseEdge{#1-v2}{#1-v6}% top arc
    \UseEdge{#1-v6}{#1-v1}% mid arc
  \end{scope}%
}

% Letter o
\newcommand{\Lettero}[2]{% prefix, baseX (full hexagon)
  \begin{scope}[x=\GlyphUnit,y=\GlyphUnit,shift={(#2,0)}]
    \HexGlyphSetup{#1}
    \UseEdge{#1-v1}{#1-v2}\UseEdge{#1-v2}{#1-v3}\UseEdge{#1-v3}{#1-v4}%
    \UseEdge{#1-v4}{#1-v5}\UseEdge{#1-v5}{#1-v6}\UseEdge{#1-v6}{#1-v1}
  \end{scope}%
}

% Letter l
\newcommand{\Letterl}[2]{% prefix, baseX
  \begin{scope}[x=\GlyphUnit,y=\GlyphUnit,shift={(#2,0)}]
    \GlyphPoint{#1}{0}{0}\GlyphPoint{#1}{0}{4}\GlyphPoint{#1}{2}{0}
    \UseEdge{#1-0-0}{#1-0-4}\UseEdge{#1-0-0}{#1-2-0}
  \end{scope}%
}

% Letter y
\newcommand{\Lettery}[2]{% prefix, baseX
  \begin{scope}[x=\GlyphUnit,y=\GlyphUnit,shift={(#2,0)}]
    \GlyphPoint{#1}{0}{3}\GlyphPoint{#1}{1}{2}\GlyphPoint{#1}{2}{3}\GlyphPoint{#1}{1}{0}
    \UseEdge{#1-0-3}{#1-1-2}\UseEdge{#1-2-3}{#1-1-2}\UseEdge{#1-1-2}{#1-1-0}
  \end{scope}%
}

% Letter G
\newcommand{\LetterG}[2]{% prefix, baseX (hex with slanted link to center)
  \begin{scope}[x=\GlyphUnit,y=\GlyphUnit,shift={(#2,0)}]
    \HexGlyphSetup{#1}
    \UseEdge{#1-v1}{#1-v2}\UseEdge{#1-v2}{#1-v3}\UseEdge{#1-v3}{#1-v4}%
    \UseEdge{#1-v4}{#1-v5}\UseEdge{#1-v5}{#1-v6}% leave v6->v1 open
    \UseEdge{#1-v6}{#1-c}% slanted to center
  \end{scope}%
}

% Letter r
\newcommand{\Letterr}[2]{% prefix, baseX (like P + down-right edge)
  \begin{scope}[x=\GlyphUnit,y=\GlyphUnit,shift={(#2,0)}]
    \HexGlyphSetup{#1}
    \UseEdge{#1-v2}{#1-v3}% left
    \UseEdge{#1-v2}{#1-v6}% top
    \UseEdge{#1-v6}{#1-v1}% mid
    \UseEdge{#1-v1}{#1-v5}% down-right leg along hex outline
  \end{scope}%
}

% Letter a
\newcommand{\Lettera}[2]{% prefix, baseX
  \begin{scope}[x=\GlyphUnit,y=\GlyphUnit,shift={(#2,0)}]
    \GlyphPoint{#1}{0}{0}\GlyphPoint{#1}{2}{0}\GlyphPoint{#1}{2}{3}\GlyphPoint{#1}{0}{3}\GlyphPoint{#1}{0}{1}\GlyphPoint{#1}{2}{1}
    \UseEdge{#1-0-0}{#1-2-0}\UseEdge{#1-2-0}{#1-2-3}\UseEdge{#1-2-3}{#1-0-3}\UseEdge{#1-0-3}{#1-0-0}
    \UseEdge{#1-0-1}{#1-2-1}
  \end{scope}%
}

% Letter p
\newcommand{\Letterp}[2]{% prefix, baseX (lowercase p, 3-stroke like P with descender)
  \begin{scope}[x=\GlyphUnit,y=\GlyphUnit,shift={(#2,0)}]
    \HexGlyphSetup{#1}
    \UseEdge{#1-v2}{#1-v4}% stem with descender
    \UseEdge{#1-v2}{#1-v6}% top arc
    \UseEdge{#1-v6}{#1-v1}% mid arc
  \end{scope}%
}

% Letter h
\newcommand{\LetterH}[2]{% prefix, baseX (capital H, hex-derived)
  \begin{scope}[x=\GlyphUnit,y=\GlyphUnit,shift={(#2,0)}]
    \HexGlyphSetup{#1}
    \UseEdge{#1-v2}{#1-v3}% left pillar
    \UseEdge{#1-v1}{#1-v4}% right pillar
    \UseEdge{#1-v6}{#1-v5}% crossbar
  \end{scope}%
}

% Letter B
\newcommand{\LetterB}[2]{% prefix, baseX
  \begin{scope}[x=\GlyphUnit,y=\GlyphUnit,shift={(#2,0)}]
    \GlyphPoint{#1}{0}{0}\GlyphPoint{#1}{0}{4}\GlyphPoint{#1}{2}{4}\GlyphPoint{#1}{2}{3}\GlyphPoint{#1}{0}{3}
    \GlyphPoint{#1}{2}{2}\GlyphPoint{#1}{2}{0}\GlyphPoint{#1}{0}{2}
    \UseEdge{#1-0-0}{#1-0-4}
    \UseEdge{#1-0-4}{#1-2-4}\UseEdge{#1-2-4}{#1-2-3}\UseEdge{#1-2-3}{#1-0-3}
    \UseEdge{#1-0-2}{#1-2-2}\UseEdge{#1-2-2}{#1-2-0}\UseEdge{#1-2-0}{#1-0-0}
  \end{scope}%
}

% Letter e
\newcommand{\Letttere}[2]{% prefix, baseX
  \begin{scope}[x=\GlyphUnit,y=\GlyphUnit,shift={(#2,0)}]
    \GlyphPoint{#1}{0}{0}\GlyphPoint{#1}{0}{3}\GlyphPoint{#1}{2}{3}\GlyphPoint{#1}{2}{0}\GlyphPoint{#1}{2}{2}\GlyphPoint{#1}{0}{2}
    \UseEdge{#1-0-3}{#1-2-3}\UseEdge{#1-0-0}{#1-2-0}\UseEdge{#1-0-0}{#1-0-3}\UseEdge{#1-0-2}{#1-2-2}
  \end{scope}%
}

% Letter n
\newcommand{\Lettern}[2]{% prefix, baseX
  \begin{scope}[x=\GlyphUnit,y=\GlyphUnit,shift={(#2,0)}]
    \GlyphPoint{#1}{0}{0}\GlyphPoint{#1}{0}{3}\GlyphPoint{#1}{2}{0}\GlyphPoint{#1}{2}{3}
    \UseEdge{#1-0-0}{#1-0-3}\UseEdge{#1-0-3}{#1-2-3}\UseEdge{#1-2-0}{#1-2-3}
  \end{scope}%
}

% Letter c
\newcommand{\Letterc}[2]{% prefix, baseX
  \begin{scope}[x=\GlyphUnit,y=\GlyphUnit,shift={(#2,0)}]
    \GlyphPoint{#1}{0}{0}\GlyphPoint{#1}{0}{3}\GlyphPoint{#1}{2}{0}\GlyphPoint{#1}{2}{3}
    \UseEdge{#1-0-0}{#1-2-0}\UseEdge{#1-0-3}{#1-2-3}\UseEdge{#1-0-0}{#1-0-3}
  \end{scope}%
}

% Letter m
\newcommand{\Letterm}[2]{% prefix, baseX (note: width 3 units)
  \begin{scope}[x=\GlyphUnit,y=\GlyphUnit,shift={(#2,0)}]
    \GlyphPoint{#1}{0}{0}\GlyphPoint{#1}{0}{3}\GlyphPoint{#1}{1}{0}\GlyphPoint{#1}{1}{3}\GlyphPoint{#1}{2}{0}\GlyphPoint{#1}{2}{3}
    \UseEdge{#1-0-0}{#1-0-3}\UseEdge{#1-1-0}{#1-1-3}\UseEdge{#1-2-0}{#1-2-3}\UseEdge{#1-0-3}{#1-1-3}\UseEdge{#1-1-3}{#1-2-3}
  \end{scope}%
}

% Letter k
\newcommand{\Letterk}[2]{% prefix, baseX (hex-derived K: stem + two diagonals)
  \begin{scope}[x=\GlyphUnit,y=\GlyphUnit,shift={(#2,0)}]
    \HexGlyphSetup{#1}
    \UseEdge{#1-v2}{#1-v4}% stem
    \UseEdge{#1-v1}{#1-v2}% upper diagonal (approx)
    \UseEdge{#1-v4}{#1-v5}% lower diagonal (approx)
  \end{scope}%
}

% Draw labels along the three straight, offset parallel edges between two nodes
% Usage: \LabelTripleEdgesTypes{nodeA}{nodeB}{labelStraight}{labelPlus}{labelMinus}
\newcommand{\LabelTripleEdgesTypes}[5]{%
  \pgfmathanglebetweenpoints{\pgfpointanchor{#1}{center}}{\pgfpointanchor{#2}{center}}%
  \let\AngleTmp\pgfmathresult%
  \begin{scope}[rotate around={\AngleTmp:(#1)}]
    \draw[draw=none] ($ (#1) + (0,0pt) $) -- node[midway,edgeLabelStraight] {#3} ($ (#2) + (0,0pt) $);
    \draw[draw=none] ($ (#1) + (0,\EdgeOffset) $) -- node[midway,edgeLabelLeft] {#4} ($ (#2) + (0,\EdgeOffset) $);
    \draw[draw=none] ($ (#1) + (0,-\EdgeOffset) $) -- node[midway,edgeLabelRight] {#5} ($ (#2) + (0,-\EdgeOffset) $);
  \end{scope}%
}
% Convenience: same label for all three parallel edges
% Usage: \LabelTripleEdges{nodeA}{nodeB}{label}
\newcommand{\LabelTripleEdges}[3]{%
  \LabelTripleEdgesTypes{#1}{#2}{#3}{#3}{#3}%
}

\begin{document}
\begin{tikzpicture}[scale=1]
  % Coordinates for center and six hexagon vertices
  \path (0,0) coordinate (c);
  \foreach \i [evaluate=\i as \ang using 90 + 60*(\i-1)] in {1,...,6} {
    \path (\ang:\HexagonRadius) coordinate (v\i);
  }

  % --- EDGES (color-specific lanes) ---
  % Straight lane drawers
  \newcommand{\DrawRedEdge}[2]{\DrawOffsetEdge{#1}{#2}{0pt}{draw=edgeAColor, line width=\EdgeThickness, line cap=round, line join=round}}
  \newcommand{\DrawGreenEdge}[2]{\DrawOffsetEdge{#1}{#2}{\EdgeOffset}{draw=edgeBColor, line width=\EdgeThickness, line cap=round, line join=round}}
  \newcommand{\DrawBlueEdge}[2]{\DrawOffsetEdge{#1}{#2}{-\EdgeOffset}{draw=edgeCColor, line width=\EdgeThickness, line cap=round, line join=round}}
  % Curved lane drawers (no endpoint offsets when curved)
  % Triple-lane outer (use \EdgeCurve)
  \newcommand{\DrawRedEdgeCurveL}[2]{\DrawCurvedEdgeL{#1}{#2}{0pt}{draw=edgeAColor, line width=\EdgeThickness, line cap=round, line join=round}}
  \newcommand{\DrawGreenEdgeCurveL}[2]{\DrawCurvedEdgeL{#1}{#2}{0pt}{draw=edgeBColor, line width=\EdgeThickness, line cap=round, line join=round}}
  \newcommand{\DrawBlueEdgeCurveR}[2]{\DrawCurvedEdgeR{#1}{#2}{0pt}{draw=edgeCColor, line width=\EdgeThickness, line cap=round, line join=round}}
  % Double-lane (use softer curvature)
  \newcommand{\DrawRedEdgeCurveLDouble}[2]{\DrawCurvedEdgeLDouble{#1}{#2}{0pt}{draw=edgeAColor, line width=\EdgeThickness, line cap=round, line join=round}}
  \newcommand{\DrawGreenEdgeCurveLDouble}[2]{\DrawCurvedEdgeLDouble{#1}{#2}{0pt}{draw=edgeBColor, line width=\EdgeThickness, line cap=round, line join=round}}
  \newcommand{\DrawGreenEdgeCurveRDouble}[2]{\DrawCurvedEdgeRDouble{#1}{#2}{0pt}{draw=edgeBColor, line width=\EdgeThickness, line cap=round, line join=round}}
  \newcommand{\DrawBlueEdgeCurveRDouble}[2]{\DrawCurvedEdgeRDouble{#1}{#2}{0pt}{draw=edgeCColor, line width=\EdgeThickness, line cap=round, line join=round}}

% --- EDGE LABELS (disabled) ---

  % === Edge sets (icon on the left) ===
  \begin{scope}[scale=\IconScale]
  % Use slightly thinner edges for the icon only (letters unchanged)
  \pgfmathsetlengthmacro{\IconEdgeThickness}{0.85*\EdgeThickness}
  % Override lane drawers locally for icon rendering
  \renewcommand{\DrawRedEdge}[2]{\DrawOffsetEdge{#1}{#2}{0pt}{draw=edgeAColor, line width=\IconEdgeThickness, line cap=round, line join=round}}
  \renewcommand{\DrawGreenEdge}[2]{\DrawOffsetEdge{#1}{#2}{\EdgeOffset}{draw=edgeBColor, line width=\IconEdgeThickness, line cap=round, line join=round}}
  \renewcommand{\DrawBlueEdge}[2]{\DrawOffsetEdge{#1}{#2}{-\EdgeOffset}{draw=edgeCColor, line width=\IconEdgeThickness, line cap=round, line join=round}}
  \renewcommand{\DrawRedEdgeCurveL}[2]{\DrawCurvedEdgeL{#1}{#2}{0pt}{draw=edgeAColor, line width=\IconEdgeThickness, line cap=round, line join=round}}
  \renewcommand{\DrawGreenEdgeCurveL}[2]{\DrawCurvedEdgeL{#1}{#2}{0pt}{draw=edgeBColor, line width=\IconEdgeThickness, line cap=round, line join=round}}
  \renewcommand{\DrawBlueEdgeCurveR}[2]{\DrawCurvedEdgeR{#1}{#2}{0pt}{draw=edgeCColor, line width=\IconEdgeThickness, line cap=round, line join=round}}
  \renewcommand{\DrawRedEdgeCurveLDouble}[2]{\DrawCurvedEdgeLDouble{#1}{#2}{0pt}{draw=edgeAColor, line width=\IconEdgeThickness, line cap=round, line join=round}}
  \renewcommand{\DrawGreenEdgeCurveLDouble}[2]{\DrawCurvedEdgeLDouble{#1}{#2}{0pt}{draw=edgeBColor, line width=\IconEdgeThickness, line cap=round, line join=round}}
  \renewcommand{\DrawGreenEdgeCurveRDouble}[2]{\DrawCurvedEdgeRDouble{#1}{#2}{0pt}{draw=edgeBColor, line width=\IconEdgeThickness, line cap=round, line join=round}}
  \renewcommand{\DrawBlueEdgeCurveRDouble}[2]{\DrawCurvedEdgeRDouble{#1}{#2}{0pt}{draw=edgeCColor, line width=\IconEdgeThickness, line cap=round, line join=round}}
  % Red edges (double-lane pairs): subtle curve
  \foreach \a/\b in {v1/c, c/v4} {\DrawRedEdgeCurveLDouble{\a}{\b};}
  % For c-v6 pair, bend opposite to green to avoid parallelism
  \DrawRedEdgeCurveLDouble{c}{v6}
  \DrawRedEdge{v6}{v1}

  % Blue edges: single-lane straight; two-lane pairs curved subtly; v6-v1 curved (outer)
  \foreach \a/\b in {v1/v2, v2/v3, v3/v4} {\DrawBlueEdge{\a}{\b};}
  \foreach \a/\b in {v4/v5, v5/c} {\DrawBlueEdgeCurveRDouble{\a}{\b};}
  \DrawBlueEdgeCurveR{v6}{v1}

  % Green edges: double-lane pairs curved subtly (opposed direction to their partner);
  %            v6-v1 triple-lane outer remains more curved
  % Use same node order as the paired lane to ensure opposing bends
  \DrawGreenEdgeCurveRDouble{c}{v4}  % oppose red c-v4 (red bends left)
  \DrawGreenEdgeCurveLDouble{v4}{v5} % oppose blue v4-v5 (blue bends right)
  \DrawGreenEdgeCurveLDouble{v5}{c}  % oppose blue v5-c (blue bends right)
  \DrawGreenEdgeCurveRDouble{c}{v6}  % oppose red c-v6 (red bends left)
  \DrawGreenEdgeCurveRDouble{v1}{c}  % oppose red v1-c (red bends left)
  \DrawGreenEdgeCurveL{v6}{v1}
  \end{scope}

  % --- NODES --- (draw only nodes that belong to edges)
  % For the main icon, draw nodes participating in edges above
  \foreach \p in {v1,c,v4,v6,v1,v2,v3,v5} {\node[vertex] at (\p) {};}

  % ===== Title to the right of the logo =====
  % Define which color lane to use for the following word
  % Red for "Poly" (baseline aligned to bottom of logo)
  \renewcommand{\UseEdge}[2]{\DrawRedEdge{#1}{#2}}
  \begin{scope}[shift={(\TextStartX,\TextBaselineY)}, x=\GlyphUnit, y=\GlyphUnit, scale=\GlyphTextScale, local bounding box=polybox]
    % P edges: 4-c, c-1, 1-6, c-6
    \HexGlyphSetup{polyP}
    \renewcommand{\UseEdge}[2]{\DrawRedEdge{#1}{#2}}
    \UseEdge{polyP-v4}{polyP-c}
    \UseEdge{polyP-c}{polyP-v1}
    \UseEdge{polyP-v1}{polyP-v6}
    \UseEdge{polyP-c}{polyP-v6}
    % Nodes used in P
    \foreach \p in {polyP-v4,polyP-c,polyP-v1,polyP-v6} {\node[vertex] at (\p) {};}
  \end{scope}
  % Set text gap to the right of custom glyph
  \pgfmathsetlengthmacro{\PolyTextGapLen}{\GapPToOly*\GlyphUnit}
  % Place remaining text for "Poly"
  \node (olytext) [anchor=base west, text=edgeAColor, inner sep=0pt, font=\LogoTextFont] at ($(polybox.east |- 0,\TextBaselineY) + (\PolyTextGapLen,0)$) {oly};

  % Blue for "Graph"
  \renewcommand{\UseEdge}[2]{\DrawBlueEdge{#1}{#2}}
  \pgfmathsetlengthmacro{\PolyGapLen}{\PolyGraphGap*\GlyphUnit}
  \begin{scope}[shift={(olytext.base east)}, xshift=\PolyGapLen, yshift=0pt, x=\GlyphUnit, y=\GlyphUnit, scale=\GlyphTextScale, local bounding box=graphbox]
    % G edges: 1-2, 2-3, 3-4, 4-5, 5-c + 1-6
    \HexGlyphSetup{graphG}
    % Draw as a single path to avoid seams/jagged joins
    \draw[draw=edgeCColor, line width=\EdgeThickness, line cap=round, line join=round]
      (graphG-v1) -- (graphG-v2) -- (graphG-v3) -- (graphG-v4) -- (graphG-v5) -- (graphG-c);
    % Separate stroke for the open segment
    \draw[draw=edgeCColor, line width=\EdgeThickness, line cap=round, line join=round]
      (graphG-v1) -- (graphG-v6);
    % Nodes used in G
    \foreach \p in {graphG-v1,graphG-v2,graphG-v3,graphG-v4,graphG-v5,graphG-v6,graphG-c} {\node[vertex] at (\p) {};}
  \end{scope}
  \pgfmathsetlengthmacro{\GraphTextGapLen}{\GapGToRaph*\GlyphUnit}
  \node[anchor=base west, text=edgeCColor, inner sep=0pt, font=\LogoTextFont] at ($(graphbox.east |- 0,\TextBaselineY) + (\GraphTextGapLen,0)$) {raph};

  % Green for "Benchmark"
  \renewcommand{\UseEdge}[2]{\DrawGreenEdge{#1}{#2}}
  \pgfmathsetlengthmacro{\GreenShift}{(4+\PolyGraphGap + 5+\GraphBenchGap)*\GlyphUnit}
  \pgfmathsetlengthmacro{\GraphGapLen}{\GraphBenchGap*\GlyphUnit}
  \begin{scope}[shift={($(graphbox.south east |- 0,\TextBaselineY)$)}, xshift=\GraphGapLen, x=\GlyphUnit, y=\GlyphUnit, scale=\GlyphTextScale, local bounding box=benchbbox]
    % B edges: partial hexagon + inner connections
    \HexGlyphSetup{benchB}
    % Center edges with green color, solid (continuous) style for letters
    \renewcommand{\UseEdge}[2]{\DrawOffsetEdge{#1}{#2}{0pt}{draw=edgeBColor, line width=\EdgeThickness}}
    % Hexagon edges (kept): 4-5, 6-1
    \UseEdge{benchB-v4}{benchB-v5}
    \UseEdge{benchB-v6}{benchB-v1}
    % Inner connections
    \UseEdge{benchB-v1}{benchB-c}
    \UseEdge{benchB-c}{benchB-v4}
    \UseEdge{benchB-c}{benchB-v5}
    \UseEdge{benchB-c}{benchB-v6}
    % Nodes used in B
    \foreach \p in {benchB-v1,benchB-v4,benchB-v5,benchB-v6,benchB-c} {\node[vertex] at (\p) {};}
  \end{scope}
  \pgfmathsetlengthmacro{\BenchTextGapLen}{\GapBToEnchmark*\GlyphUnit}
  \node[anchor=base west, text=edgeBColor, inner sep=0pt, font=\LogoTextFont] at ($(benchbbox.east |- 0,\TextBaselineY) + (\BenchTextGapLen,0)$) {enchmark};
\end{tikzpicture}
\end{document}


